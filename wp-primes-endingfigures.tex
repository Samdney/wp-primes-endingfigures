\documentclass{amsart}
\usepackage{footnote}
\usepackage{amssymb}
\usepackage{url}
%\usepackage{graphicx}

% ======================================================================
\begin{document}
% ======================================================================
\title[NOTES: Primes - Ending figures]{NOTES: Primes\\
- \\
Ending figures of primes and not primes}

\author{Carolin Z\"obelein}
\urladdr{http://www.carolin-zoebelein.de}
\email{contact@carolin-zoebelein.de, PGP: D4A7 35E8 D47F 801F 2CF6 2BA7 927A FD3C DE47 E13B}
\thanks{wp-primes-endingfigures (old name: notes00000primes00000), CC BY-ND 3.0 DE}

\subjclass[2010]{Primary 11N05}
\date{October 9, 2015}

\dedicatory{My notes are only sketches. Not final results! They are to be a basis for common discussion in terms of community based research.}
% ======================================================================
\begin{abstract}
	Disproof of the proposition that all numbers with certain ending figures are always primes. (Twitter response)
\end{abstract}

\maketitle
% ======================================================================
% Introduction
\section{Introduction}
\label{s:introduction}
% ----------------------------------------------------------------------
At 2015/09/24 @NatalieCogan posted the following Tweet on Twitter: \url{https://twitter.com/NatalieCogan/status/647123082595794944}\\
Content: "All numbers that end with $1$, $7$, $3$ or $9$ and aren't divisible by $3$ is a prime number (except for $91=7\cdot 13$)."\\
After some requests I will show how you can easy disproof this proposition.
% ======================================================================
% Repetition: Integer divisible numbers
\section{Repetition: Integer divisible numbers}
\label{s:repetition}
% ----------------------------------------------------------------------
In the following a very short repetition about integer divisble numbers. Apart from $2$ and $3$ all prime numbers can be written as (set $O_{-}$) $p_{-} = 6\gamma_{-} - 1$ or (set $O_{+}$) $p_{+} = 6\gamma_{+} + 1$, $\gamma \in \mathbb{N}$. But of course not all of this numbers are prime numbers like, for example, $6\cdot4 + 1 = 25 = 5 \cdot 5$. We are able (see \cite{CaZoeb}) to describe all integer divisible numbers of the set $O_{-}$ by the equation
\begin{equation}
	\gamma_{-} = 6\alpha\beta + \alpha - \beta
\label{eq:set-_eq+-}\end{equation}
and all integer divisible numbers of the set $O_{+}$ by the equations
\begin{equation}
	\gamma_{+,1} = 6\alpha\beta - \alpha - \beta
\label{eq:set+_eq--}\end{equation}
and
\begin{equation}
	\gamma_{+,2} = 6\alpha\beta + \alpha + \beta,
\label{eq:set+_eq++}\end{equation}
$\alpha, \beta \in \mathbb{N}$. With this we are able to disproof the proposition.
% ======================================================================
% Disproof of proposition
\section{Disproof of proposition}
\label{s:disproof}
% ----------------------------------------------------------------------
We start with $p = 6\gamma + 1$. For all $\gamma = 5n$, $n \in \mathbb{N}$, we receive all $p$-values which ends with "$0$", since $p = 6 \cdot 5 n + 1 = 30n + 1$. So we look at $p = 6\gamma + c$, $\gamma = 5n$, $c \in [0, 9]$ and will discuss all of this cases.
\begin{itemize}
	\item Case $c=0,2,4,6,8$: $\Rightarrow$ $p$ is always even and integer divisible by $2$!
	\item Case $c=3, 9$: $\Rightarrow$ $p$ is always odd and integer divisible by $3$!
	\item Case $c=1$: $\Rightarrow$ We have our starting case $p = 6\cdot5n + 1$	
	\item Case $c=5$: $\Rightarrow$ We have $p = 6\cdot5n + 5 = 6\cdot5n + 6 - 1 = 6\left(5n + 1\right) - 1$	
	\item Case $c=7$: $\Rightarrow$ We have $p = 6\cdot5n + 7 = 6\cdot5n + 6 + 1 = 6\left(5n + 1\right) + 1$
\end{itemize}
Now we will make the disproof of the cases $c=1$,$5$ and $7$.\\
Be $c=1$:\\
We take equation (\ref{eq:set+_eq++}) and have to show that exists solutions for
\[ 5n = 6\alpha\beta + \alpha + \beta. \]
This is, for example, easy fulfilled for $\alpha = 5a$ and $\beta = 5b$, $a,b \in \mathbb{N}$.
\begin{flushright}$\Box$\end{flushright}
Be $c=5$:\\
We take equation (\ref{eq:set-_eq+-})
\begin{equation*}\begin{split}
	5n + 1 & = 6\alpha\beta + \alpha - \beta\\
	5n & = 6\alpha\beta + \alpha - \beta - 1.
\end{split}\label{eq:c5}\end{equation*}
This is, for example, easy fulfilled for $\alpha = 5a$ and $\beta = 5b - 1$, $a,b \in \mathbb{N}$.
\begin{flushright}$\Box$\end{flushright}
Be $c=7$:\\
We also take equation (\ref{eq:set+_eq++})
\begin{equation*}\begin{split}
	5n + 1 & = 6\alpha\beta + \alpha + \beta\\
	5n & = 6\alpha\beta + \alpha + \beta - 1
\end{split}\label{eq:c7}\end{equation*}
This is, for example, easy fulfilled for $\alpha = 5a$ and $\beta = 5b + 1$, $a,b \in \mathbb{N}$.
\begin{flushright}$\Box$\end{flushright}
% ======================================================================
% Bibliography
\bibliographystyle{amsplain}
\bibliography{wp-primes-endingfigures}
% ======================================================================
\end{document}
